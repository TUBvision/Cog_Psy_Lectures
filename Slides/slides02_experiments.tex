\documentclass[]{beamer}
% \geometry{papersize={16cm,9.60cm}}
\usepackage{etex}
\usepackage{amsmath}
\usepackage{tikz}
\usepackage{multimedia}
\usetheme{Boadilla}
\usepackage{graphicx}
%\usepackage{inputenc}

% \mode<presentation>
% {
%   \usetheme{default}
%   \setbeamercovered{transparent}
% }


% {\vskip5pt}

%% customize layout, bullet points navigation toolbar
\setbeamertemplate{navigation symbols}{}%remove navigation symbols
\setbeamertemplate{enumerate items}[default]
\setbeamertemplate{navigation symbols}{}
\setbeamertemplate{itemize items}[circle]
\setbeamercolor{enumerate item}{fg=black}

\setbeamertemplate{footline}{}
\setbeamersize{text margin left = 2.0em}
\setbeamersize{text margin right = 2.0em}

\usepackage{times}
\usepackage[T1]{fontenc}

% Or whatever. Note that the encoding and the font should match. If T1
% does not look nice, try deleting the line with the fontenc.

\setbeamertemplate{navigation symbols}{}

\title{ Cognitive (Neuro) Psychology }
\subtitle{II. Experiments in Psychology}
\author{ Marianne Maertens }
\institute[TU Berlin]{Technische Universit\"at Berlin}
\date{July 2016}

\begin{document}
\setbeamertemplate{enumerate items}[default]
\setbeamertemplate{headline}

\frame{\titlepage}

\AtBeginSection[]
{
  \begin{frame}<beamer>
    \frametitle{Overview}
    \tableofcontents[currentsection]
  \end{frame}
}


\begin{frame}
 \frametitle{Overview}
 \begin{itemize}
  \item Psychology as a scientific discipline
  \item Variables
  \item Hypotheses
  \item Components of an experiment
  \item Practical steps in an experiment
 \end{itemize}
\end{frame}




\section{Psychology as a scientific discipline}
\begin{frame}
 \frametitle{Common sense psychology ...}
contains a number of statements which are not being tested with respect to their validity:\\

  \begin{itemize}
   \item birds of a feather flock together
   \item actions speak louder than words
   \item can't judge a book by its cover
   \item ... 
   \item German products are usually high quality ... (VW)
   \item German trains are on time ...
  \end{itemize}
\end{frame}


\begin{frame}
 \frametitle{Common sense psychology}
\begin{columns}[c]
 \column[c]{10cm}
  \begin{itemize}
 \setlength{\itemsep}{5pt}
   \item objects fall down if you release the grip
   \item birds fly, foxes don't
   \item[] ...
   \item<2->[$\rightarrow$] basic understanding of concepts in physics, biology, psychology, etc. ... without studying them \item<2-> \alert{folk-}physics, psychology, biology, ...
  \end{itemize}
 \column{2cm}
  \includegraphics[width=20mm]{../../../figures/huber_1.png}
\end{columns}

\vspace{4mm}
 \only<3->{\begin{block}{Common sense is ...}
   \begin{itemize}
    \item<4-> mix of true and false beliefs
    \item<4-> prejudices and stereotypes
    \item<4-> full of contradictions 
    \item<5->[!] common sense wisdom does not undergo critical testing with \textcolor{blue}{scientific methods}
   \end{itemize}
  \end{block}}
\end{frame}

\begin{frame}{Why don't we notice the inadequacy of folk wisdom?}
\begin{overlayarea}{110mm}{70mm}
  \begin{enumerate}[<+->]
\setlength{\itemsep}{10pt}
   \item Even behavior based on false presumptions, might still
lead to desired consequences out of pure luck.
   \item Folk wisdom might influence our behavior so that we act in a
certain way and this causes the desired outcome.
   \item The way we interpret and remember events is influenced by our expectations. We selectively attend to evidence in favour of our beliefs. 
   
\only<3>{
\begin{center}
\includegraphics[width=40mm]{../../../figures/huber_folk_theology.png}
\end{center}}

   \item The knowledge content of folk psychology is used to explain events
\textbf{post-hoc}. Plausibility of the explanation is more important than accuracy. 
  \end{enumerate}
\end{overlayarea}
\end{frame}




\begin{frame}{Perceptual vs cognitive biases}
\begin{center}
\includegraphics<1>[width=60mm]{../../../figures/bregman_Bs.png}
\includegraphics<2>[width=20mm]{../../../figures/banana_penetrates_brick.png}
\end{center}
\end{frame}

\begin{frame}{Cognitive biases in science}
\begin{center}
\includegraphics<1>[width=65mm]{../../../figures/reproducibility_nature15_fallacies.jpg}
\includegraphics<2>[width=65mm]{../../../figures/reproducibility_nature15.jpg}
\end{center}
\end{frame}

\begin{frame}{The scientific method}
 \begin{itemize}
   \item How to distinguish science from non-science? \item[] \textcolor{blue}{The demarcation problem}
   \item When is one theory better than another?
  \end{itemize}

\vspace{2mm}
\only<2->{
\textbf{Philosophy of science - 3 main schools of thought}
 \begin{enumerate}
   \item militant positivism: program to find definition that puts every theory 
in its proper place
   \item scepticism, cultural relativism: demarcation problem is unsolvable, 
because there is no demarcation line, no progress but changing fashions 
   \item elitist authoritarianism: there is a demarcation line, but there are 
no demarcation criteria, belief in a wise judge (great scientist)
  \end{enumerate}
}
\end{frame}


\begin{frame}{The scientific method}
\textbf{Mertonian norms of science (R. Merton, 1910-2003) }
 \begin{enumerate}
   \item[] \textbf{C}ommunalism: common ownership of scientific ideas
   \item[] \textbf{U}niversalism: claims to truth are evaluated in terms of 
universal or impersonal criteria 
   \item[] \textbf{D}esinterestedness: scientists are rewarded for acting in 
ways that outardly appear to be selfless
   \item[] \textbf{O}rganized scepticism: all ideas must be tested and are 
subject to rigorous, structured community scrutiny 
  \end{enumerate}
\end{frame}

\begin{frame}
\begin{figure}
\begin{center}
   \includegraphics[width=90mm]{../../../figures/scientist_fool_nature15.png}

   \includegraphics<1>[width=60mm]{../../../figures/scientist_fool_nature15_picture.png}
\end{center}
\end{figure}
\end{frame}


\begin{frame}{The scientific method}
\textit{\textquotedblleft As a researcher, I'm not trying to produce misleading results, but I do have a stake in the outcome.\textquotedblright \ And that 
gives the mind excellent motivation to find what it is primed to find.} (Nuzzo, 2015)
\end{frame}



\section{Variables}
\begin{frame}
 \frametitle{Goals of scientific research}
\begin{columns}[T]
\begin{column}{60mm}
Accumulation of knowledge
\begin{figure}
\includegraphics[width=50mm]{../../../figures/stars.jpg} 
\end{figure}
\end{column}
\begin{column}{60mm}
Investigation of lawful relations
\begin{figure}
\includegraphics[width=50mm]{../../../figures/huber_money_subjvalue.png}
\end{figure}
\end{column}
\end{columns}
\end{frame}

\begin{frame}
\frametitle{Variables}
\begin{itemize}
 \item concept to describe characteristic attributes of human beings, animals, objects, systems etc...
 \item[]
 \item <2-> take at least 2 different values, but only one at a time, \\ 
e.g. fear of spiders - yes or no; emotions: joy, sadness, fear, shame, curiosity; intelligence: 70, ...145
 \item<2-> varying levels of abstraction: age vs. political attitudes
 \item<2-> more or less directly observable: soup intake vs. intelligence
 \item<2->[$\rightarrow$] \textbf{operationalization}
\end{itemize}
\end{frame}


\begin{frame}
 
\begin{exampleblock}{How to select variables?}
Write down all the variables that would allow a complete description of yourself in the current situation! 
\end{exampleblock}
\end{frame}

\begin{frame}
 \frametitle{How to select variables}
\begin{itemize}[<+->]
 \item []\textbf{Appearance:} shoe size, hair colour, form of nostrils, ...
 \item []\textbf{Social relations:} son/daughter, mother/father, personally
acquainted with, friend of, contemporary of, neighbour of, boss of, ...
 \item []\textbf{Personality attributes:} neurotic, cooperative, curious, ...
 \item []\textbf{Goals/desires/preferences:} artistic, work-related,
study-related, personality-wise, ...
 \item []\textbf{Physiological Variables:} blood pressure, heart rate, size of
liver, ...
 \item []\textbf{Individual history:} childhood memories, family situation, ...
 \item []\textbf{Knowledge:} German grammar, memory of locations, soccer
results,
cooking skills, ...
\end{itemize}
\end{frame}


\begin{frame}{Variable selection}
\begin{itemize}
 \item []theoretical account influences variables under study
\end{itemize}

\onslide<2->{

\textbf{Example: Theoretical accounts of depression}
\begin{columns}[T]
\begin{column}{8cm}
\begin{itemize}[<+->]
 \item Hippocrates - black bile
 \item middle ages - obsessed by the devil
 \item characteristic hand lines (H\"oping, 1689)
 \item psychoanalysis - early childhood experiences
 \item cognitive theory - thinking patterns (Beck)
 \item theory of learned helplessness (Seligman)
 \item neurotransmitter imbalance
\end{itemize}
\end{column}
 
\begin{column}{3cm}
\only<4>{
\begin{center}
\includegraphics[width=30mm]{../../../figures/huber_handlines.png}
\end{center}}
\end{column}
\end{columns}
}
\end{frame}


\begin{frame}
\begin{Large}
\textbf{Important part of psychological research is:}
\vspace{5mm}
\end{Large}
\begin{itemize}
\setlength{\itemsep}{10pt}
 \item to identify variables that are crucial for answering a certain
question 
 \item to assign observable variables to theoretically interesting variables 
\end{itemize}
\end{frame}


\section{Hypotheses}
\begin{frame}{Scientific research starts with questions}
\begin{itemize}
\setlength{\itemsep}{5pt}
 \item Is intelligence inherited?
 \item Is therapy x more effective than therapy y to cure disorder Z?
 \item What are the factors that influence whether a person is attracted to
another or not?
 \item Under which conditions do humans behave aggressively? 
 \item At what age do kids have an understanding or the concept of
'probability'?
 \item How do people in their 30ies think about death?
\end{itemize}
\end{frame}

\begin{frame}{What is a hypothesis?}

\begin{itemize}
 \item<2-> A hypothesis is a presumed answer to a question.
 \item<3-> To test whether or not a hypothesis applies it needs to be exposed to scientific testing. 
 \item<4-> The basic idea of empirical testing is the derivation of a prediction from the hypothesis and its test under realistic conditions.
 \item<5-> [!] empirical hypotheses allow predictions about and \textcolor{blue}{comparison with reality}
\end{itemize}
\end{frame}


\begin{frame}
\begin{overlayarea}{110mm}{80mm}
 \textcolor{blue}{Example:}
\begin{itemize}
 \item [If] aggressive behavior is learned by imitation,
 \item [then] the observation of an aggressive model should increase the probability that a person will act aggressively herself (Model learning, learning by imitation).
\end{itemize}

\begin{center}
\includegraphics[width=60mm]{../../../figures/huber_modellearning.png}
\end{center}

\only<2->{
\textcolor{blue}{Empirical hypothesis} 
\begin{itemize}
\item Guests in a restaurant who observe an aggressive model are more likely to react aggressively in response to a cold soup than guests who do not observe an aggressive model.
\end{itemize}
}
\end{overlayarea}
\end{frame}



\begin{frame}
 \frametitle{Formation of hypotheses}
 \includegraphics<1>[width=80mm]{../../../figures/huber_hypothesis.png}
 \includegraphics<2>[width=80mm]{../../../figures/huber_explorative.png}
\end{frame}
 

\begin{frame}
 \frametitle{Testing of hypotheses}
\begin{itemize}
 \item I am deeply convinced, that ...
 \item reference to authorities 
 \item 'proof' by example
 \item[!]\textcolor{blue}{prone to cognitive biases}
 \item[]
 \item<2-> The power of examples can be explained by the observation, that the subjective probability for an event is influenced by the ease of which we can find examples for it in our memory
 \item<2->[=] \textcolor{blue}{availability heuristic} (Tversky \& Kahneman, 1973)
\end{itemize}
\end{frame}


\begin{frame}
\frametitle{Preconditions for hypothesis testing} 
\begin{overlayarea}{110mm}{80mm}
\begin{columns}[T]
\begin{column}{60mm}
\begin{enumerate}[<+->]
\setlength{\itemsep}{10pt}
 \item consistency
 \item falsifiability
 \item operationalizability
 \item generation \textbf{before} the test
\end{enumerate}
\end{column}
\begin{column}{50mm}
\includegraphics<4>[width=50mm]{../../../figures/huber_robin_hood.png} 
\end{column}
\end{columns}
\end{overlayarea}
\end{frame}


\section{Experiments}
\begin{frame}
 \frametitle{Experimental ingredients}
\begin{enumerate}[<+->]
 \item experimental vs. non-experimental research (observation/correlational)
 \item types of variables
 \begin{itemize}
  \item independent variable
  \item dependent variable
  \item confounding variables
  \item moderating variables
 \end{itemize}
 \item experimenter vs. subjects/participants/observers
 \item control of confounds
  \begin{itemize}
   \item constant value
   \item random variation
  \end{itemize}
 \item classification of experiments
  \begin{itemize}
   \item number of IV: one-way, two-way, ...
   \item number of DV: univariate, multivariate, ...
   \item assignment of subjects: between, within, mixed designs 
   \item laboratory or field studies 
   \item longitudinal or cross-sectional
   \item ... 
  \end{itemize}
\end{enumerate}
\end{frame}


\begin{frame}
\frametitle{Research Question}
\begin{itemize}
 \item unanswered question: \textit{What are the causes for aggressive
behavior?}
 \item ... studying the literature ...
 \item[$\Rightarrow$] Does an aggressive exemplar (role model) influence
aggressive behavior? 
\end{itemize}
\end{frame}

\begin{frame}
\frametitle{Hypotheses}
\textbf{If} a person \textit{A} observes the aggressive behavior
of person \textit{B} in a certain situation, \textbf{then} this increases the
probability that also person \textit{A} will act aggressively in that
situation.
\begin{itemize}
 \item[IV:] observation or non-observation of aggressive model
 \item[DV:] aggressive behavior of person A
\end{itemize}
\end{frame}

% \begin{frame}
% \frametitle{Operationalization}
% \begin{center}
% \begin{Large}
% \vspace{5mm}
% Please find in the next 5 min as many indicators as possible for the state of
% hunger!
% \end{Large}
% \end{center}
% \end{frame}

% \begin{frame}
% \frametitle{Operationalization}
% \begin{itemize}
%  \item time since last meal
%  \item blood sugar level 
%  \item amount of food eaten
%  \item speed of food intake
%  \item running speed while running to food source
%  \item registration of stomache contractions
%  \item asking
%  \item what negative consequences are accepted to gain access to food
%  \item what positive consequences are denied to gain access to food
% \end{itemize}
% \end{frame}

\begin{frame}
\begin{exampleblock}{Operationalization}
Write down indicators of aggressive behavior!
\end{exampleblock}
\end{frame}

\begin{frame}
\frametitle{Operationalization}
\begin{itemize}
\item loudness of voice
\item adrenaline level
\item heart rate
\item verbal statements
\item body posture
\item interpersonal distance
\item physical attacks
\item report 
\end{itemize}
\end{frame}

% 
% \begin{frame}
% \frametitle{Operationalization}
% \begin{itemize}
%  \item assign observable phenomena to the concepts of the hypothesis
%  \item easy to realize
%  \item comparably new to all participants
% \end{itemize}
% \begin{exampleblock}{children at computer}
% \begin{itemize}
%  \item[IV-1]: children observe a model that yells at or hits a computer
% every time that something goes wrong
%  \item[IV-2]: children observe a model that behaves neutral towards
% the computer
%  \item both situations could be videotaped and played to the children while
% waiting for the 'real' experiment
%  \item[DV:] children have to interact with the computer and the computer will
% produce errors from time to time
% \end{itemize}
% \end{exampleblock}
% \end{frame}
% 


\begin{frame}
\frametitle{Goodness of operationalization}
\begin{itemize}
 \item background knowledge helps
 \item a theoretical context helps even more
 \item critical discussion is minimum 
 \item different ways to operationalize one and the same variable/concepts usually indicates necessity to redefine the concept - lack of \textcolor{blue}{construct validity}
 \item the more abstract the construct of interest the more challenging the operationalization e.g. consciousness 
\end{itemize}
\end{frame}

% 
% \begin{frame}
% \frametitle{2. Techniques to collect data}
% \begin{columns}[c]
%  \column{9cm}
% \begin{itemize}
%  \item behavioral observation e.g. case study, groups
%  \item survey, e.g. interview, questionnaire
%  \item experiment vs. correlation (cause and effect vs. relationships)
%  \item analysis of behavioral traces, artifacts
% \end{itemize}
%  \column{4cm}
% \includegraphics<1>[width=30mm]{../../../figures/huber_observation.png}
% \includegraphics<2>[width=35mm]{../../../figures/huber_survey.png}
% \includegraphics<3>[width=35mm]{../../../figures/huber_test.png}
% \includegraphics<4>[width=35mm]{../../../figures/huber_traces.png}
% \end{columns}
% \end{frame}


\begin{frame}
\frametitle{Measurement}

\begin{itemize}
 \item assign numbers to measurement items (e.g. to be measured
individuals, objects, events) such that characteristic empirical relations between the measurement items are represented by the corresponding numerical relations in the numbers 
\item resulting numbers are called scale values
\item[]
\item<2->[$\Rightarrow$] Why cant we just describe the variation verbally?
\item<2->[] 
\item<3-> assignment of the right scale: nominal, ordinal, interval,
ratio, absolute
\item<3-> validity and reliability, ... 
\end{itemize}
\end{frame}


\begin{frame}
\frametitle{Experimental design}
\begin{itemize}
 \item logical structure of the experiment
 \item e.g. 1 independent variable with 2 levels
\end{itemize}

\only<2->{
\begin{tabular}[t]{c|c|c}
   & \textbf{time 1} & \textbf{time 2}  \\ \hline
   \textbf{group 1} & IV - level 1 & DV\\
   \hline
   \textbf{group 2} & IV - level 2 & DV\\
\end{tabular}}

\vspace{2mm}

\only<3>{
\begin{tabular}[t]{c|c|c}
   & \textbf{time 1} & \textbf{time 2}  \\ \hline
   \textbf{group 1} & aggressive model & loudness of voice\\
   \hline
   \textbf{group 2} & neutral model & loudness of voice\\
\end{tabular}}

\vspace{2mm}

\only<4->{
\begin{tabular}[t]{c|c|c|c}
   & \textbf{time 1} & \textbf{time 2} & \textbf{time 3}  \\
   & pre-test &  &  post-test \\ \hline
   \textbf{group 1} & DV & IV - level 1  & DV\\
   \hline
   \textbf{group 2} & DV & IV - level 2  & DV\\
\end{tabular}}

\vspace{2mm}

\only<5->{
\begin{tabular}[t]{c|c|c|c|c}
   & \textbf{t1} & \textbf{t2} & \textbf{t3} & \textbf{t3} \\ \hline
\textbf{1 group} & IV - level 1     & DV          & IV - level 2 &  DV\\
\end{tabular}}
\end{frame}


\begin{frame}
\frametitle{Control of confounding variables}
\begin{overlayarea}{110mm}{80mm}
\textcolor{blue}{Factors of the participants:}
\begin{itemize}
 \item e.g. gender, age, intelligence,... 
 \item [$\Rightarrow$] \textbf{Matching} (variable needs to be known and
measurable, small groups)
 \item e.g. intelligence, motivation, mood, ... 
 \item [$\Rightarrow$] \textbf{Randomization}
\end{itemize}
\vspace{3mm}

\only<2->{
\textcolor{blue}{Factors of the experimental setup:}
\begin{itemize}
  \item Noise, experimenter effects, ...
  \item [$\Rightarrow$] \textbf{Elimination}
  \item time of the day, number of sessions per week, ...
  \item [$\Rightarrow$] \textbf{Constancy}
  \item [$\Rightarrow$] \textbf{Random variation} 
  \item [$\Rightarrow$] \textbf{Control group}
\end{itemize}
}
\end{overlayarea}
\end{frame}


\begin{frame}
\frametitle{Control of confounding variables}
\textcolor{blue}{... in the example:}
\begin{itemize}
 \item degree of identification with the model
 \item aggressive tendencies
 \item behavior of the experimenter
 \item time of the day, weather, differently pleasant rooms, ...
\end{itemize}
\begin{columns}[T]
 \begin{column}{5cm}
   \begin{center} \textbf{constant value} \end{center}
 \begin{itemize}
  \item videotaping the model
  \item written instructions
  \item identical rooms
 \end{itemize}
 \end{column}

 \begin{column}{5cm}
\begin{center}\textbf{ random variation}\end{center}
 \begin{itemize}
  \item random assignment of observers to groups (\alert{sampling})
 \end{itemize}
 \end{column}
\end{columns}
\end{frame}



\begin{frame}
\frametitle{Summary}
\includegraphics[width=80mm]{../../../figures/huber_summary.png}
\end{frame}

\begin{frame}
 \frametitle{Reference}
\begin{columns}[T]
 \begin{column}{60mm}
This lecture is based on the following book: \\

Oswald Huber (2009). Das psychologische Experiment: Eine Einf\"uhrung. Bern:
Huber.
 \end{column}
 \begin{column}{40mm}
\includegraphics[width=40mm]{../../../figures/huber_einband.png}
 \end{column}
\end{columns}
\end{frame}

\end{document}

