\documentclass[]{beamer}
% \geometry{papersize={16cm,9.60cm}}
\usepackage{etex}
\usepackage{amsmath}
\usepackage{tikz}
\usepackage{multimedia}
\usetheme{Boadilla}
\usepackage{graphicx}
\usepackage{url}
%\usepackage{inputenc}

% \mode<presentation>
% {
%   \usetheme{default}
%   \setbeamercovered{transparent}
% }


% {\vskip5pt}

%% customize layout, bullet points navigation toolbar
\setbeamertemplate{navigation symbols}{}%remove navigation symbols
\setbeamertemplate{enumerate items}[default]
\setbeamertemplate{navigation symbols}{}
\setbeamertemplate{itemize items}[circle]
\setbeamercolor{enumerate item}{fg=black}

\setbeamertemplate{footline}{}
\setbeamersize{text margin left = 2.0em}
\setbeamersize{text margin right = 2.0em}

\usepackage{times}
\usepackage[T1]{fontenc}

% Or whatever. Note that the encoding and the font should match. If T1
% does not look nice, try deleting the line with the fontenc.

\setbeamertemplate{navigation symbols}{}

\title{ Cognitive (Neuro) Psychology }
\subtitle{VIII. Attention}
\author{ Marianne Maertens }
\institute[TU Berlin]{Technische Universit\"at Berlin}
\date{September 2016}

\begin{document}
\setbeamertemplate{enumerate items}[default]
\setbeamertemplate{headline}

\frame{\titlepage}

\AtBeginSection[]
{
  \begin{frame}<beamer>
    \frametitle{Layout}
    \tableofcontents[currentsection]
  \end{frame}
}


\begin{frame}
 \begin{center}
\includegraphics<1>[width=50mm]{figs/l8/serial_reading.png}
\includegraphics<2->[width=100mm]{figs/l8/waldo.png}
 \end{center} 
\end{frame}


\begin{frame}
 \frametitle{Attention}
\begin{overlayarea}{110mm}{80mm}
 \begin{itemize}
  \item a very large set of selective processes in the brain
 \item[] 
 \item<2-> impossible to handle all inputs at once
 \item<2-> nervous system has evolved mechanisms that are able to restrict processing to a subset of things, places, ideas, moments in time
 \end{itemize}
\only<3->{
\textcolor{blue}{Selective Attention}
 \begin{itemize}
  \item form of attention involved when processing is restricted to a subset of possible stimuli
\end{itemize}
}
 \begin{center}
 \end{center} 
\end{overlayarea}
 \end{frame}

\begin{frame}
\frametitle{Selective Attention}
 \begin{center}
\includegraphics[width=35mm]{figs/l1/dichotic_listening_task.png}
 \end{center} 
\end{frame}


\begin{frame}
 \frametitle{Overview}
\begin{itemize}[<+->]
  \setlength{\itemsep}{5pt}
 \item Experimental paradigms for studying attention
 \begin{itemize}
  \item in space
  \item in time
 \end{itemize}
 \item Physiological basis of attention
 \item Disorders of attention
 \item Scene perception
\end{itemize}
\end{frame}

\begin{frame}
\frametitle{Attention}
\begin{overlayarea}{110mm}{80mm}
\begin{itemize}
 \item<1-> not a single \textit{thing} and does not have a single locus in the brain
 \item[]
 \item<2-> external vs. internal
 \item<3-> overt vs. covert
 \item<4-> focussed vs. divided
 \item<5-> sustained
 \item<6-> selective
\end{itemize}
\end{overlayarea}
\end{frame}


\begin{frame}
 \frametitle{Selection in space}
\begin{overlayarea}{130mm}{75mm}
\only<1->{
\begin{itemize}
 \item[] What does it mean to ``pay attention``?
\end{itemize}
}
\only<2->{
\begin{center}
\textcolor{blue}{Cueing paradigm} \begin{scriptsize}(Posner, 1980) \end{scriptsize}

\includegraphics[width=80mm]{figs/l8/cueing_basic.png}
\end{center}
}

\only<3->{
Hit a response key as fast as possible when the probe appears!\\
\textbf{Dependent variable:} reaction time (RT)
}
\end{overlayarea}
\end{frame}



\begin{frame}
 \frametitle{Selection in space}
\begin{overlayarea}{110mm}{75mm}

\begin{center}
\textcolor{blue}{Cueing paradigm} \begin{scriptsize}(Posner, 1980) \end{scriptsize}
\end{center}
\begin{columns}[T]
 \begin{column}{50mm}
\centering valid
\includegraphics[width=60mm]{figs/l8/cueing_valid.png}
 \end{column}

 \begin{column}{80mm}
\centering invalid
 
\includegraphics[width=60mm]{figs/l8/cueing_invalid.png}
 \end{column}
\end{columns}

\vspace{5mm}
\begin{itemize}
 \item<2->[Cue] stimulus that indicates where (or what) subsequent stimulus might be. Valid (giving correct information), invalid or neutral.
 \item<3-> exogeneous/peripheral vs. endogeneous/symbolic cues
 \item<4->[SOA] stimulus onset asynchrony  - time between onset of one stimulus and onset of another.
\end{itemize}
\end{overlayarea}
\end{frame}


\begin{frame}
\frametitle{Selection in scace}
\begin{overlayarea}{110mm}{75mm}
\begin{center}
\textcolor{blue}{Cueing paradigm} \begin{scriptsize}(Posner, 1980) \end{scriptsize}
\end{center}

 \begin{center}
\includegraphics<1-2>[width=70mm]{figs/l8/cueing_results.png}
 \end{center}
\begin{itemize}
 \item benefit from valid cue: $RT_{invalid} - RT_{valid}$
 \item increases with processing time for cue 
 \item peripheral cue is processed quicker
\end{itemize}
\end{overlayarea}
\end{frame}


\begin{frame}
 \frametitle{Visual search}
\begin{overlayarea}{130mm}{80mm}
first described by Werner Reichardt (1950)
 \begin{center}
\includegraphics[width=60mm]{figs/l7/reichardt_detector_manycells.png}
 \end{center}
\only<2->{
\begin{itemize}
 \item direction-selective
 \item tuned to velocity 
 \item fast adaptation prevents response to large objects
\end{itemize}
}
\end{overlayarea}
\end{frame}



\begin{frame}

\begin{block}{Activity}
Explore the motion detection circuit with the activity on the following webpage \url{http://sites.sinauer.com/wolfe4e/wa08.01.html}
\end{block}
\end{frame}



\begin{frame}
 \frametitle{Apparent motion}
 \begin{overlayarea}{110mm}{80mm}
\begin{itemize}
   \item Reichardt detector does not require continuous motion, responds equally to object that appears in A's receptive field, disappears and reappears in B's receptive field
\end{itemize}

\begin{columns}[T]
 \begin{column}{40mm}
\includegraphics<1->[width=30mm]{figs/l7/reichardt_detector_single.png}
 \end{column}

 \begin{column}{40mm}
\begin{center}
\includegraphics<2>[width=40mm]{figs/l7/daumenkino_daffy_1.png} 
\includegraphics<3>[width=40mm]{figs/l7/daumenkino_daffy_2.png} 
\includegraphics<4>[width=40mm]{figs/l7/daumenkino_daffy_3.png} 
\includegraphics<5>[width=40mm]{figs/l7/daumenkino_daffy_4.png} 
\end{center}
\end{column}
\end{columns}
\end{overlayarea}
\end{frame}


\begin{frame}
\frametitle{Apparent motion}
 \begin{itemize}
 \item illusory impression of smooth motion resulting from rapid alternation of objects that appear in different locations in rapid succession
 \item stop-motion
\end{itemize}
\begin{center}
\includegraphics[width=70mm]{figs/l7/filming_shaun2.jpg} 
\end{center}
\end{frame}

\begin{frame}
 \frametitle{The correspondence problem}
\begin{overlayarea}{110mm}{80mm}
 \begin{columns}[T]
 \begin{column}{40mm}
\includegraphics<1->[width=35mm]{figs/l7/correspondence_1.png}
 \end{column}

 \begin{column}{40mm}
\includegraphics<1->[width=35mm]{figs/l7/correspondence_2.png} 
\end{column}
\end{columns}

\only<2->{
\vspace{3mm}
How does the visual system know which circles in frame 2 correspond to which circles in frame 1?}
\begin{center}
\includegraphics<2>[width=40mm]{figs/l7/correspondence_3.png} 
\end{center}
\end{overlayarea}
\end{frame}

\begin{frame}
 \frametitle{The aperture problem}
\begin{overlayarea}{120mm}{70mm}
when a moving object is viewed through an aperture, the direction of motion of a local part of the object may be ambiguous 
\begin{center}
\includegraphics[width=80mm]{figs/l7/aperture.png} 
\end{center}

\only<2->{
\begin{itemize}
 \item Why is such an artificial situation relevant at all?
 \item<3->[$\rightarrow$] every V1 cell sees the world through a small aperture
\end{itemize}
}
\end{overlayarea}
\end{frame}


\begin{frame}
 \frametitle{Local vs. global motion}
\begin{overlayarea}{110mm}{70mm}
\begin{columns}[T]
\begin{column}{70mm}
\includegraphics[width=70mm]{figs/l7/global_motion.png} 
\end{column}

\begin{column}{50mm}
\only<2->{
\begin{itemize}
 \item none of the V1 cells can tell with certainty which visual elements correspond to one another when an object moves
 \item<3->[$\rightarrow$] have another set of neurons 'listen' to the V1 neurons and integrate the potentially conflicting signals 
\end{itemize}
}
\end{column}
 \end{columns}
\end{overlayarea}
\end{frame}

\begin{frame}
 \frametitle{Where in the brain are global motion detectors?}
\begin{center}
\includegraphics<1>[width=90mm]{figs/l7/cortical_pathways_motion.png} 
\includegraphics<2->[width=70mm]{figs/l7/MT_brain.png}  
\end{center}
\only<2->{
\begin{itemize}
 \item area MT in non-human primates - middle temporal area
 \item human equivalent of MT localized with fMRI MT+/V5
 \item<3-> MT cells are selective for motions in one particular direction
 \item<4->[$\rightarrow$] \textit{Do MT cells correspond to global motion detector or are they also local motion detectors?}
\end{itemize}
}
\end{frame}


\begin{frame}
 \frametitle{Newsome and Pare (1988) paradigm}
\begin{overlayarea}{110mm}{70mm}
\begin{center}
\includegraphics<1->[width=90mm]{figs/l7/newsome_pare.png} 
\end{center}

\begin{itemize}
 \item monkeys were trained in motion discrimination in correlated dot displays
 \item<2-> no single dot sufficient to signal overall direction of motion
 \item<2->[$\rightarrow$] to detect correlated direction a neuron must integrate information from many locations
\end{itemize}
\end{overlayarea}
\end{frame}


\begin{frame}
 \frametitle{Newsome and Pare (1988) paradigm}
\begin{overlayarea}{110mm}{70mm}
\begin{center}
\includegraphics<1->[width=90mm]{figs/l7/newsome_pare.png} 
\end{center}

\begin{itemize}
 \item after training monkeys recognized motion direction when only 2-3\% od the dots moved coherently
 \item<2-> experimental \textcolor{blue}{lesion} of area MT
 \item<3-> following surgery discrimination thresholds were 10 times as high: 20\% correlated dots
 \item<3-> ability to discriminate orientation of stationary patterns was generally unimpaired
\end{itemize}
\end{overlayarea}
\end{frame}

\begin{frame}
 \frametitle{Salzman, Britten \& Newsome, 1990}
 \begin{itemize}
  \item lesions may be incomplete or may affect other structures
  \item[]
  \item<2-> new group of monkeys was trained on motion discrimination
  \item<2-> recording of neurons to find groups of neurons that responded to one particular direction of motion
 \item[]
 \item<3-> stimulation with leftward motion while electrically stimulating neurons that respond to rightward motion
 \item<3->[?]
 \item<4-> monkeys reported motion in the stimulated neuron's preferred direction  
 \end{itemize}
\end{frame}

\begin{frame}
 \frametitle{Motion aftereffects (MAE)}
\begin{overlayarea}{110mm}{70mm}
 \begin{itemize}
  \item converging evidence for motion perception in humans?
 \item<2-> MAE: illusion of motion of a stationary object that occurs after prolonged exposure to a moving object
 \end{itemize}

\only<3->{
\begin{block}{Activity}
Explore the motion aftereffect with the activity on the following webpage \url{http://sites.sinauer.com/wolfe4e/wa08.04.html}
\begin{itemize}
 \item What happens when you adapt to downward motion and then look at something moving horizontally? Make a prediction and then test your hypothesis!
 \item Adapt for 15 s ro rightward motion with your right eye open and left eye closed. After adaptation quickly switch your eyes when looking at the test. 
\end{itemize}

 \end{block}
}
\end{overlayarea}
\end{frame}


\begin{frame}
 \frametitle{}
\begin{overlayarea}{110mm}{70mm}
\begin{columns}[T]
\begin{column}{70mm}
\end{column}

\begin{column}{50mm}
\end{column}
 \end{columns}
\end{overlayarea}
\end{frame}




\begin{frame}
 \frametitle{References}
\begin{small}
\begin{itemize}
 \item  Wolfe, J.M., Kluender, K.R. \& Levi, D.M. (2012).\textit{Sensation \& Perception}. Sinauer Associates: Sunderland, MA. 
\end{itemize}
\end{small}
\end{frame}


\end{document}