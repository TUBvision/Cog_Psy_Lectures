\documentclass[]{beamer}
% \geometry{papersize={16cm,9.60cm}}
\usepackage{etex}
\usepackage{amsmath}
\usepackage{tikz}
\usepackage{multimedia}
\usetheme{Boadilla}
\usepackage{graphicx}
%\usepackage{inputenc}

% \mode<presentation>
% {
%   \usetheme{default}
%   \setbeamercovered{transparent}
% }


% {\vskip5pt}

%% customize layout, bullet points navigation toolbar
\setbeamertemplate{navigation symbols}{}%remove navigation symbols
\setbeamertemplate{enumerate items}[default]
\setbeamertemplate{navigation symbols}{}
\setbeamertemplate{itemize items}[circle]
\setbeamercolor{enumerate item}{fg=black}

\setbeamertemplate{footline}{}
\setbeamersize{text margin left = 2.0em}
\setbeamersize{text margin right = 2.0em}

\usepackage{times}
\usepackage[T1]{fontenc}

% Or whatever. Note that the encoding and the font should match. If T1
% does not look nice, try deleting the line with the fontenc.

\setbeamertemplate{navigation symbols}{}

\title{ Cognitive (Neuro) Psychology }
\subtitle{IV. Psychophysics}
\author{ Marianne Maertens }
\institute[TU Berlin]{Technische Universit\"at Berlin}
\date{July 2016}

\begin{document}
\setbeamertemplate{enumerate items}[default]
\setbeamertemplate{headline}

\frame{\titlepage}

\AtBeginSection[]
{
  \begin{frame}<beamer>
    \frametitle{Layout}
    \tableofcontents[currentsection]
  \end{frame}
}


\begin{frame}
 \frametitle{What is real?}
\begin{columns}[T]
\begin{column}{60mm}
``What is real? How do you define real? If you're talking about what you can feel, what you can smell, what you can taste and see, then real is simply electrical signals interpreted by your brain. This is the world that you know.'' - Morpheus' answer to Neo in \textit{The Matrix}, 1999
\end{column}
\begin{column}{50mm}
\includegraphics<1>[width=45mm]{figs/l4/matrix_real.png} 
\end{column}
\end{columns}
\end{frame}



\begin{frame}
 \frametitle{Psychophysics}
\begin{center}
\includegraphics<1->[width=50mm]{../../../figures/pmf.png} 
\end{center}

\begin{itemize}
\setlength{\itemsep}{5pt}
 \item<2-> subdiscipline of psychology
 \item<3-> addresses the relationship between physical stimuli, $x$, and their subjective correlates (percepts), $\Psi(x)$ 
\end{itemize}
\end{frame}


\begin{frame}
 \frametitle{Psychophysics}

\begin{columns}[T]
\begin{column}{60mm}
\includegraphics<1>[width=45mm]{figs/l4/soundwave.png} 

\includegraphics<1>[width=45mm]{figs/l4/smell.jpg} 

\end{column}
\begin{column}{50mm}
\includegraphics<1>[width=45mm]{figs/l4/wavelength.jpg} 

\includegraphics<1>[width=40mm]{figs/l4/fire_stim.png} 
\end{column}
\end{columns}
\end{frame}

\begin{frame}
\frametitle{Outline}
 
\end{frame}


\begin{frame}
 \frametitle{Components of a psychophysics experiment}
\begin{itemize}
\setlength{\itemsep}{5pt}
 \item stimuli
 \item task
 \item method
 \item analysis
 \item measure
\end{itemize}
\end{frame}


\begin{frame}
 \frametitle{Measure: contrast sensitivity}

\begin{center}
\includegraphics<1>[width=100mm]{../../../figures/CSF_wiki.png} 
\end{center}
\end{frame}


\begin{frame}
 \frametitle{Stimulus}
\begin{center}
\includegraphics<1>[width=100mm]{../../../figures/weber_contrast_stimulus.png} 
\end{center}
\end{frame}


\begin{frame}
 \frametitle{Task}
\begin{overlayarea}{110mm}{80mm}
\begin{columns}[T]
 \begin{column}{50mm}
\begin{itemize}
\setlength{\itemsep}{10pt}
\item[]
\item[] 
\item Method of adjustment
\item[]
\item[]
\item[]
\item<2-> 2-alternative forced-coice (2-AFC)
\end{itemize}
 \end{column}

\begin{column}{60mm}
\begin{center}
\includegraphics<1->[width=50mm]{../../../figures/weber_adjustment.png} 

\vspace{15mm}

\includegraphics<2->[width=50mm]{../../../figures/weber_2afc.png} 
\end{center}
 \end{column}
\end{columns}
\end{overlayarea}
\end{frame}



\begin{frame}
 \frametitle{Method \& Analysis}
\begin{overlayarea}{110mm}{80mm}
\begin{columns}[T]
 \begin{column}{50mm}
\begin{itemize}
\setlength{\itemsep}{10pt}
\item[]
\item Adaptive procedure
\item[]
\item[]
\item<2-> Method of constant stimuli
\end{itemize}
 \end{column}

\begin{column}{60mm}
\begin{center}
\includegraphics<1->[width=50mm]{../../../figures/weber_adaptive.png} 

\vspace{5mm}

\includegraphics<2->[width=50mm]{../../../figures/weber_constant_method.png} 
\end{center}
 \end{column}
\end{columns}
\end{overlayarea}
\end{frame}


\begin{frame}
 \frametitle{Dichotomies}
\begin{center}
 \includegraphics[width=100mm]{../../../figures/cloud_types.png}
\end{center}

\begin{itemize}
 \item all understanding begins with making comparisons and those comparisons, in turn, lead to the construction of categories
 \item[]
 \item<2-> simplify and make explicit design choices
 \item<2-> which method is appropriate for studying which aspect of visual functioning 
\end{itemize}
\end{frame}


\begin{frame}
\frametitle{Objective vs. Subjective}

\begin{itemize}
 \item nature of measurement e.g. Muller-Lyer illusion
\end{itemize}

\begin{overlayarea}{110mm}{70mm}
\begin{columns}[T]
 \begin{column}{50mm}
\begin{center}
\includegraphics<1->[width=30mm]{../../../figures/muller_lyer.png} 
\end{center}
\begin{itemize}
 \item<3> PSE: point of subjective equality
 \item<3> objective: correct vs. incorrect
\end{itemize}
 \end{column}

\begin{column}{60mm}
\begin{center}
\includegraphics<2>[width=50mm]{../../../figures/muller_lyer_pmf.png} 
\includegraphics<3->[width=50mm]{../../../figures/muller_lyer_pmf_pse.png} 
\end{center}
 \end{column}
\end{columns}
\end{overlayarea}
\end{frame}

\begin{frame}
\frametitle{Objective vs. Subjective}

\begin{itemize}
 \item method of data collection
 \item[]
 \item[] method of adjustment = \textcolor{blue}{subjective}
 \item[] method of constant stimuli = \textcolor{blue}{objective}
\end{itemize}
\end{frame}


\begin{frame}
\frametitle{Performance vs. appearance}
\begin{overlayarea}{110mm}{80mm}
\begin{columns}[T]
 \begin{column}{55mm}
\begin{itemize}
 \item How good is an observer in a particular task? e.g. orientation discrimination in the fovea vs. the periphery
\end{itemize}
\begin{center}
\includegraphics<2->[width=30mm]{../../../figures/weber_ori_discrim.png} 
\end{center}
\begin{itemize}
 \item<3> accuracy
 \item<3> threshold
\end{itemize}
 \end{column}

 \begin{column}{55mm}
  \begin{itemize}
   \item Performance can not be meaningfully considered as 'better' e.g. apparent lightness of targets
  \end{itemize}
\begin{center}
\includegraphics<2->[width=30mm]{../../../figures/weber_lightness_appear.png} 
\end{center}
\begin{itemize}
 \item<3> PSE
 \item<3> scales
\end{itemize}
 \end{column}
\end{columns}

\end{overlayarea}
\end{frame}


\begin{frame}
\frametitle{Forced-choice vs. non-forced}
\begin{itemize}
 \item criterion-free vs. criterion-dependent
\end{itemize}

\end{frame}

\begin{frame}
\frametitle{Detection vs. Discrimination}

\begin{overlayarea}{110mm}{80mm}
\begin{columns}[T]
 \begin{column}{55mm}
\begin{itemize}
 \item is there a / where is the single stimulus?
\end{itemize}
\begin{center}
\includegraphics<2->[width=30mm]{../../../figures/weber_detection.png} 
\end{center}
 \end{column}

 \begin{column}{55mm}
  \begin{itemize}
   \item which of two stimuli is more 'x'?
  \end{itemize}
\begin{center}
\includegraphics<2->[width=30mm]{../../../figures/weber_discrimination.png} 
\end{center}
 \end{column}
\end{columns}
\vspace{5mm}
\begin{itemize}
 \item<3> contrast discrimination - detection of an increment on a pedestal
\end{itemize}
\end{overlayarea}
\end{frame}




\begin{frame}
 \frametitle{Psychophysics Summary}
\begin{center}
\includegraphics[width=110mm]{../../../figures/psychophysics_classification.png} 
\end{center}
\end{frame}



\begin{frame}

\begin{center}
\includegraphics[width=60mm]{../../../figures/adelson_demo_bar.png} 
\end{center}

 \begin{exampleblock}{Thinking}
The two checks indicated by arrows have the same retinal luminance but differ in apparent lightness. Design an experiment to quantify the perceived difference between the two checks!
 \end{exampleblock}
\end{frame}


\begin{frame}
 \frametitle{Psychometric function theories: high threshold theory}

\begin{center}
\includegraphics[width=60mm]{../../../figures/weber_2ifc.png} 
\end{center}

\begin{itemize}
 \item 2-IFC: $S$ signal and $N$ noise 
 \item Which of the two intervals contained the stimulus?
 \item sensory evidence fluctuates from trial to trial $n \sim N(0,1)$
\end{itemize}
\end{frame}

\begin{frame}
 \frametitle{Psychometric function theories: high threshold theory}

\begin{center}
\includegraphics<1>[width=90mm]{../../../figures/signal_pmf_0_5.png} 
\includegraphics<2>[width=90mm]{../../../figures/signal_pmf_1_5.png} 
\includegraphics<3>[width=90mm]{../../../figures/signal_pmf_3_0.png} 
\includegraphics<4>[width=90mm]{../../../figures/signal_pmf_4_5.png} 
\end{center}
\begin{itemize}
 \item according to high-threshold theory the sensory mechanism will detect the stimulus when the amount of sensory evidence exceeds a fixed internal criterion
 \item $F(x)$: probability that the threshold will be exceeded by a stimulus of intensity $x$
\end{itemize}
\end{frame}

\begin{frame}
 \frametitle{High threshold theory - decision process}
\begin{center}
\includegraphics<1>[width=90mm]{../../../figures/high_threshold_decision.png} 
\end{center}
\begin{itemize}
%  \item $S_{I2}$ > $t$ $\Rightarrow$ stimulus was in that interval
\item relation between observable behavior $\Psi(x)$ and unobservable decision mechanism $F(x)$
% \item various serious of events that lead to correct and incorrect responses
\item $\lambda:$ lapse rate
\item $\gamma:$ guess rate
\end{itemize}
\end{frame}


\begin{frame}
 \frametitle{High threshold theory - assumptions}
\begin{itemize}
\item amount of sensory evidence accumulated is unavailable to the decision process
\item probability that the threshold is exceeded when x=0, i.e. by noise is effectively zero, no false alarms
\end{itemize}
\end{frame}



\begin{frame}
 \frametitle{References}
\begin{small}
\begin{itemize}
 \item  Kingdom \& Prins, Psychophysics. A practical introduction. 
 \item 
 \item 
\end{itemize}
\end{small}
\end{frame}


\end{document}