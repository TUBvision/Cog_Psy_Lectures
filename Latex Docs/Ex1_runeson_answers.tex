\documentclass[12pt,english]{scrartcl}

\usepackage{amsmath,amssymb}
%\usepackage[amssymb]{SIunits}
\usepackage{babel}
\usepackage[latin1]{inputenc}
\usepackage{graphicx}
\usepackage{color}
\usepackage{url}


\title{KOGW-PM-KNP:  Exercise 1 - Runeson's Planimeter}
\author{}
\date{\today}

\begin{document}

\maketitle

\begin{enumerate}
 \item What is the purpose of a polar planimeter? Read the respective paragraph and watch the video on the following website \url{https://www.youtube.com/watch?v=7R07IWiXV1g}. Provide a basic description of the functioning principle of the planimeter.
  \begin{itemize}
   \color{blue} 
   \item measure the area of irregular shapes
   \item measures area by exploiting the mechanics of the device, produces correct output via correct use of the instrument without necessity to calculate anything
  \end{itemize}

 \item Explain, in your own words, the difference between smart and rote instruments.
  \begin{itemize}
   \color{blue} 
   \item smart mechanisms are designed to solve a particular task like a script
   \item rote mechanisms are designed in a modular way to solve more complex tasks by cleverly combining the modules, libraries consisting of functions
  \end{itemize}

 \item What are, according to Runeson, the two reasons why perception might use smart mechanisms. 
  \begin{itemize}
   \color{blue} 
   \item the to-be processed information and environment are stable 
   \item design principle of minimal complexity
  \end{itemize}

 \item Describe the task given to the PWP (person with a planimeter) by the SPP (sensory psychophysicist). How does PWP solve the task? What conclusion does the SPP draw and what goes wrong? 
  \begin{itemize}
   \color{blue} 
   \item measure the length of a line
   \item measure area of circle that has roughly the given line length as radius and computes the line length according to the formula $A=\pi*r^2$ $r = \sqrt{\frac{A}{\pi}}$
   \item variable answers for such a simple task, planimeter is crude and unreliable instrument
  \end{itemize}

 \item Describe the task given to the PWP by the CP (cognitive  psychologist). How does PWP solve the task? What conclusion does the CP draw and what goes wrong?
  \begin{itemize}
   \color{blue} 
   \item measure area of an irregular shape - think protocol, introspection!
   \item measures area directly
   \item discrepancy between accuracy of result and lack of conscious reporting - thesis on unconscious inference
  \end{itemize}
 
 \item Explain, in your own words, the difference between perception and cognition.
  \begin{itemize}
   \color{blue} 
   \item perception: processes not accessible to introspection
   \item cognition: conscious percepts of thought processes, reportable
  \end{itemize}

 \item What do you learn from this?

\end{enumerate}


% \begin{enumerate}
%  \item Explain, in your own words, the difference between perception and cognition, using the concept of a "smart" mechanism [1]? Begin with a basic description of how the planimeter works [1]. Include an example of a discovered smart mechanism [1].
%   \begin{itemize}
%   \color{blue} 
%   \item The planimeter acts as a transducer between intention (wanting to know the area of a irregular shape) and a physically "complex" variable (the area) without direct knowledge of the underlying mathematical principles of its construction. 
%   \item No calculations are explicitly made, rather the mechanics of the device produces the desired output via correct use of the instrument. In a similar way, our brains can access complex variables of the physical environment (e.g. color, sound, temperature) without us needing to calculate anything.  This is perception as Runeson describes it.
%   \item Tennis player watching a ball need only "know" the rate of image expansion to calculate time-to-collision, rather than compute all the myriad physical variables. P.176 Runeson Paper - Lee 1974. Allow other good examples of smart mechanisms.
%   \end{itemize}
%   
%  \item Name two properties that Runeson identifies as definitive for smart mechanisms [2]?
%   \color{blue}
%   \item[]
%   Any 2 from below:
%   \begin{itemize}
%   \item Stable 
%   \item Continuous 
%   \item Efficient / Simplicity
%   \item Not influenced by Cognitive effects (mood, fatigue, drugs, etc.)
%   \end{itemize}
%  
%  \color{black}
%  \item What is the "principle of equal simplicity" [2]?
%    \color{blue}
%    \item[]
%    The principle of simplicity enables us to reach conclusions about the operation of a system [1] by comparing variables which quantify that function [1] (e.g. speed, simplicity, variance) p.175 Runeson Paper - Sensory Psychophysics
%  
%  \color{black}
%  \item Name two areas of psychology (besides perception) in which smart mechanisms apply [2]:
%  \item[]
%  \color{blue}
%  Any 2 from below:
%  \begin{itemize}
%   \item Developmental psychology
%   \item Learning
%   \item Attention
%   \item Mastery
%  \end{itemize}
% 
%  \color{black}
%  \item Invent a smart mechanism that could improve on our everyday perception [1].
%  \item[]
%  \color{blue}
%  Anything that follows the vague concept of a planimeter whilst improving our perceptual abilities e.g. : 
%   \begin{itemize}
%    \item In-built statistical evaluator, allowing for better decision making when presented with large or misleading information
%    \item Taste buds that give a breakdown of nutritional content of food, so we can choose healthier options. 
%   \end{itemize}
% 
% \end{enumerate}

\end{document}