\documentclass[12pt,english]{scrartcl}

\usepackage{amsmath,amssymb}
%\usepackage[amssymb]{SIunits}
\usepackage{babel}
\usepackage[latin1]{inputenc}
\usepackage{graphicx}
\usepackage{color}

\title{KOGW-PM-KNP: Tutorial 8 Answers - Eye movement analysis}
\author{}
\date{\today}

\begin{document}

\maketitle

\begin{enumerate}
 \item Using planimeter, measure the shape provided. Describe how you did it [1]. What are the qualities of such a measurement [1], compared to say: counting squares*.
 \item[]
 \color{blue}
 Trace the edge of the shape, read off the value [1] 
 \item[]
 Any 1 from:
 \begin{itemize}
 \item More efficient
 \item Higher precision
 \item No calculations necessary 
 \end{itemize} 
 
 \color{black}
 \item Now try to measure the length of a line with the planimeter. Compare this with using a ruler. Can you think of a way in which you could measure the length of a line using the planimeter, given that it only measures area [1]?
 \item[]
 \color{blue}
 Use the area of a known shape to calculate the length (e.g. circle $Area =/pi radius^2 $). [1]
 
 \color{black}
 \item Consider what methods you employed to solve Task 2. How does this differ from Task 1 [1]? How does this compare to perception vs cognition [1]?
 \item[]
 \color{blue}
 You employed a method to solve it rather than just following a instruction. [1] \\
 Perception as directly accessing complex data, cognition as calculating complex data. [1]
 
 \color{black}
 \item Discuss with your group possible way to differentiate between pseudo-perceptual judgements and true perceptual reports? Use the table below to help, adding any more you can think of [3]. What are the problems with using such distinctions[2]?
 \item[]

  \begin{center}
   \begin{tabular}{ || c c c || } 
   \hline
   Measurement Quality & Perception & Cognition \\ 
   \hline
   Precision & \color{blue} High & \color{blue} Low  \\ 
   \hline
   Stability & \color{blue} High & \color{blue} Low  \\ 
   \hline
   Time taken  & \color{blue} Low  & \color{blue} High \\
   \hline
   \end{tabular}
  \end{center}
  \item[]
  
 \color{blue}
 Pitfalls [2]
 \begin{itemize}
 \item Brain functions follow a spectrum (between perception and cognition) with unknown cutoffs between.
 \item We can't directly access much of our own internal functioning.
 \item Participants verbal reports may be biased.
 \item Accurate cognitive compensation could appear to be perceptual.
 \item Other good comments.
 \end{itemize}


 \end{enumerate}

 Maximum marks 10/10 \\
 * Counting the number of squares, of known area, within the perimeter of the shape.
\end{document}