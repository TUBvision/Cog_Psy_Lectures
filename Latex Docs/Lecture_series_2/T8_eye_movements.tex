\documentclass[12pt,english]{scrartcl}

\usepackage{amsmath,amssymb}
%\usepackage[amssymb]{SIunits}
\usepackage{babel}
\usepackage[latin1]{inputenc}
\usepackage{graphicx}
\usepackage{color}

\title{KOGW-PM-KNP: Tutorial 8 Answers - Eye movement analysis}
\author{}
\date{\today}

\begin{document}
	
	\maketitle
	In this exercise you will be working through Python code (step\_by\_step\_quant.py) provided to you. The bulk of the code is broken up into auxiliary (aux\_funcs.py) and quantification (quant\_funcs.py) functions, stored in their respective scripts. You may views these functions for additional understanding but the focus here is on the concepts discussed in the following questions. \\
	
	The experiment we follow is eye-tracking through an object search task. Namely, looking for the odd-one-out in a selection of images. The eye movements are sampled every few milliseconds, which can then be analysed, giving us insight into what is important when undertaking a visual search, particularly what parts we fixate upon.
	
	
	\begin{enumerate}
		\item We first want to get familiar with the data we are using. Look through the code and find where 'bdata' and 'edata' are imported. Type 'bdata.keys()' to view the different labels of data within the bdata dictionary. Next, plot the outputs for nitems = 4,8 and 12. Also change the presence of the target and see if you can observe a difference [1]. \\
		
		\color{blue}
		The stimuli with more items have a more varied tracking path [1] 
		
		\color{black}
		\item Having seen the raw data we want to extract fixation points from the tracking path. One method to do this is via velocity discrimination. Given that we know the time interval between each point on the path, can you think of a way of calculating the velocity between each point[1]? \\
		
		\color{blue}
		Velocity = distance / time . Therefore we can calculate the velocity by measuring the distance between subsequent points along the tracking path. \\
		
		\color{black}
		Follow the code through, plotting the various methods applied to the raw data to extract fixation points, via velocity discrimination. The essential point is that below a certain velocity value, data points are defined as fixations, above the threshold they are defined as saccades. See what happens when you change the velocity threshold value \\
		
		\item Run the code for dispersion based quantification. Note the difference in quality between this method and the previous. 
		\item[]
		\color{blue}
		Dispersion correctly isolated more fixation points.
		
		\color{black}
		\item TBC
		
		
	\end{enumerate}
	
	Maximum marks 10/10 \\
	
\end{document}