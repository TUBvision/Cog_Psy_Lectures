\documentclass[12pt,english]{scrartcl}

\usepackage{amsmath,amssymb}
%\usepackage[amssymb]{SIunits}
\usepackage{babel}
\usepackage[latin1]{inputenc}
\usepackage{graphicx}
\usepackage{url}

\title{KOGW-PM-KNP:  Exercise 1 Questions - Runeson`s Planimeter}
\author{}
\date{\today}

\begin{document}

% \maketitle
\begin{center}
\textbf{\begin{LARGE}KOGW-PM-KNP:\\ \vspace{3mm}  Exercise 1 - Runeson's smart mechanisms                                                   \end{LARGE}}
\end{center}

\raggedright
The mainstream approach to the study of perception and cognition uses the analogy to information processing. Runeson suggests an alternative conception and by doing that he introduces a distinction between perception and cognition and at the same time treats important problems of the experimental study of both domains. He does that by using the metaphor of a polar planimeter. I think this is an interesting metaphor and would like to share it with you. To facilitate comprehension of the article please answer the following questions. You are encouraged to work in pairs but please do so quietly.

Read the paper on smart perceptual mechanisms by Runeson starting on page 173 with the last paragraph before \textit{The polar planimeter}. ``Now, if the theory ...'' and finishing on page 177 before \textit{Perceptual development}.\\


\begin{enumerate}
 \item What is the purpose of a polar planimeter? Read the respective paragraph and watch the video on the following website \url{https://www.youtube.com/watch?v=7R07IWiXV1g}. Provide a basic description of the functioning principle of the planimeter.

 \item Explain, in your own words, the difference between smart and rote instruments.

 \item What are, according to Runeson, the two reasons why perception might use smart mechanisms. 

 \item Describe the task given to the PWP (person with a planimeter) by the SPP (sensory psychophysicist). How does PWP solve the task? What conclusion does the SPP draw and what goes wrong?
 
 \item Describe the task given to the PWP by the CP (cognitive  psychologist). How does PWP solve the task? What conclusion does the CP draw and what goes wrong?

 \item Explain, in your own words, the difference between perception and cognition.

 \item What do you learn from this? 
\end{enumerate}

\end{document}