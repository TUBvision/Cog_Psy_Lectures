\documentclass[12pt,english]{scrartcl}

\usepackage{amsmath,amssymb}
%\usepackage[amssymb]{SIunits}
\usepackage{babel}
\usepackage[latin1]{inputenc}
\usepackage{graphicx}
\usepackage{color}

\title{KOGW-PM-KNP: Tutorial 2 - G{\"u}nt{\"u}rk{\"u}n`s Magpies}
\author{}
\date{\today}

\begin{document}

% \maketitle
\begin{center}
\textbf{\begin{LARGE}KOGW-PM-KNP: Tutorial 5 - Quiroga
 \end{LARGE}}
\end{center}


\begin{enumerate}
 \item What do the authors mean by \textit{invariance}. Explain the concept using the Figure as an example. \\
 \color{blue}
 \begin{itemize}
 \item MTL neurons represent high-level information in an abstract manner.
 \item independent of metric characteristics of the image, different view points, even letters
 \item in the figure the neuron responds to faces and is invariant to differences in the species, or degree of naturalness
 \end{itemize}
 
  
 \color{black}
 \item Describe the testing procedure used in the study.\\
 \color{blue}
 \begin{itemize}
  \item patients with implanted depth electrodes to treat intractable epilepsy
  \item screening: interview, responses to set of images of famous persons, landmarks, etc ... 
  \item testing: images with strongest response in screening were shown from distinct views  
  \item task: press a button to indicate whether a face was present
  \item responsive units: cells that had a response of more than the baseline mean $+$5sd
  \end{itemize}

 
 \color{black}
 \item Describe the main result.\\
 \color{blue}
 \begin{itemize}
  \item cells in hippocampus, parahippocampal gyrus, amygdala and entorhinal cortex respond specifically to images of a person or landmark alone but not to similar pictures  
 \end{itemize}
 

  
 \color{black}
 \item What potential confound do the authors identify and how do they rule it out? \\
 \color{blue}
 \begin{itemize}
  \item Movement artefacts. Differences in timing.
 \end{itemize}
 
\color{black}
 \item What are the two extreme alternatives for how the brain could represent objects? What are benefits and drawbacks of each alternative? Can you think of a better third alternative? \\
 \color{blue}
 \begin{itemize}
  \item highly selective neurons (conserve energy, combinatorial problem, vulnerable)
  \item distributed population of neurons (protect against information loss, generalize categorize - novel similar stimulus - partial activation)
 \item[$rightarrow$] sparse distributed coding such that activity in several neurons is required to represent a stimulus (e.g. a particular face). 
 \end{itemize}
 
\color{black}
 \item Is there an alternative explanation for the finding instead of single neurons being the neural substrate that recognizes grandma?\\
 \color{blue}
 \begin{itemize}
 \item Cells might be a link between visual processes and stored memories. Represent HOW an object is recognized.
 \end{itemize}
\end{enumerate}

\end{document}