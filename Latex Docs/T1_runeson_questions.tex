\documentclass[12pt,english]{scrartcl}

\usepackage{amsmath,amssymb}
%\usepackage[amssymb]{SIunits}
\usepackage{babel}
\usepackage[latin1]{inputenc}
\usepackage{graphicx}
\usepackage{color}

\title{KOGW-PM-KNP: Tutorial 1 Questions- Runeson`s Planimeter}
\author{}
\date{\today}

\begin{document}

\maketitle

\begin{enumerate}
 \item Using planimeter, measure the shape provided. Describe how you did it [1]. What are the qualities of such a measurement [1], compared to say: counting squares*.
 \item[]
 \item Now try to measure the length of a line with the planimeter. Compare this with using a ruler. Can you think of a way in which you could measure the length of a line using the planimeter, given that it only measures area [1]?
 \item[]
 \item Consider what methods you employed to solve Task 2. How does this differ from Task 1 [1]? How does this compare to perception vs cognition [1]?
 \item[]
 \item Discuss with your group possible way to differentiate between pseudo-perceptual judgements and true perceptual reports? Use the table below to help, adding any more you can think of [3]. What are the problems with using such distinctions[2]?
 \item[]

  \begin{center}
   \begin{tabular}{ || c c c || } 
   \hline
   Measurement Quality & Perception & Cognition \\ 
   \hline
   Precision &  &  \\ 
   \hline
   Stability &  &   \\ 
   \hline
   Time taken  &  &  \\
   \hline
   \end{tabular}
  \end{center}
  \item[]

 \end{enumerate}

 Maximum marks 10/10
 * Counting the number of squares, of known area, within the perimeter of the shape.
\end{document}