\documentclass[12pt,english]{scrartcl}

\usepackage{amsmath,amssymb}
%\usepackage[amssymb]{SIunits}
\usepackage{babel}
\usepackage[latin1]{inputenc}
\usepackage{graphicx}
\usepackage{color}
\usepackage{url}

\title{KOGW-PM-KNP: Tutorial 2 - G{\"u}nt{\"u}rk{\"u}n's Magpies}
% \author{}
% \date{\today}

\begin{document}

% \maketitle
\begin{center}
\textbf{\begin{LARGE}KOGW-PM-KNP:\\ \vspace{3mm} Tutorial 2 - G{\"u}nt{\"u}rk{\"u}n's Magpies                                                                           \end{LARGE}}
\end{center}
Studying intelligence is difficult in humans but even more so in other animals. We do not know what intelligence means neither in human nor in non-human creatures. That has lead researchers in the past to assume rather little intelligent behavior in animals. However, there is accumulating that this was a misjudgment on behalf of the human. Look at this video for example to get a different idea of how smart 'even' birds  can be 
\url{http://www.dailymail.co.uk/sciencetech/article-2590046/Crows-intelligent-CHILDREN-Study-reveals-birds-intelligence-seven-year-old.html}.

However, studying intelligent behavior in animals is difficult. Onur G{\"u}nt{\"u}rk{\"u}n and Alex Kacelnik are some of the pioneers in doing that. Today you will read a paper by Prior, Schwarz and G{\"u}nt{\"u}rk{\"u}n (2008) to get an idea of how intelligent behavior can  be scientifically studied in one representative of the animal kingdom. You will use your newly acquired knowledge about experiment design to comprehend and evaluate their work. 

Please read the paper and - working in pairs - answer the following questions:
\begin{enumerate}
 \item What was the research question the authors wanted to address?
 \item What experimental hypothesis did they formulate in order to test their research questions? Try to express this hypothesis in ``IF ..., then ...'' form.
 \item Identify the independent and dependent variables in the experiment. 
 \item What type of realizations/operationalizations did the authors use to observe the dependent variable?
 \item What variables did the authors identify as potentially confounding factors and how did they control for them? 
 \item What other potential confounds could you think of that the authors did not control and that could potentially invalidate their research?
 \item Label the design as within- or between group design.
 \item Create an example design matrix, for one subject, that is, a matrix containing one row for each trial of the experiment, and one column each for the independent variable, the control variables and the trial number.
 \end{enumerate}


\end{document}