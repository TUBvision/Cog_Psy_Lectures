\documentclass[12pt,english]{scrartcl}

\usepackage{amsmath,amssymb}
%\usepackage[amssymb]{SIunits}
\usepackage{babel}
\usepackage[latin1]{inputenc}
\usepackage{graphicx}
\usepackage{color}
\usepackage[font={color=blue},figurename=Fig.,labelfont={it}]{caption}

\title{KOGW-PM-KNP: The Psychometric Function in R}

\begin{document}

\maketitle


Complete the tasks using R. Your answers should include the code you wrote and relevant output images.

\section*{Task 0. R}
Explore the help facilities of R. At the command line type \texttt{help(help)} and read over the documentation. Try the commands \texttt{help.start()}, \texttt{demo(graphics)}.

\section*{Task 1. Fitting a Psychometric Function to data}



\section*{Task 2. Maximum Likelihood Criterion}
To introduce the concept of 'likelihood' we start with a simple 1-Parameter example. 
Imagine we have a coin and wish to estimate the parameter corresponding to the probability that our coin lands 'heads' on any given flip of the coin. We designate the parameter $alpha$. We perform the experiment of flipping the coin 10 times. After each flip we denote whether it landed heads (H) or tails (T). The results are respectively: \\
HHTHTTHHTH\\

The likelihood function associated with our parameter of interest is:\\

$\displaystyle L(a|\textbf{y}) = \prod_{k=1}^n p(y_k|a)$

where $a$ is a potential value for our parameter $\alpha$, $p(y_k|a)$ is the probability of observing outcome $y$ on trial $k$, given or assuming value $a$ for $\alpha$ and $n$ is the total number of trials. In our example it is obvious, that $p(y_k=H) = a$ and $p(y_k=T)=1-a$. The equation utilizes the multiplicative rule in probability theory for independent random events. 

\begin{enumerate}
 \item Calculate the likelihood for a=0.4.
 \item Plot $L(a|\textbf{y})$ as a function of $a$ across the range $0\leq a \leq 1$
 \item As the term implies, the maximum likelihood estimte of parameter $\alpha$ is the value of $a$ that maximizes the likelihood function $L(a|\textbf{y})$. For which $a$ is $L(a|\textbf{y})$ maximal in the present example.
 \item Please comment on the difference between a likelihood and a probability.   
\end{enumerate}


  

\end{document}