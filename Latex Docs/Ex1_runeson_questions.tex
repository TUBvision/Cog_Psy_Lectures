\documentclass[12pt,english]{scrartcl}

\usepackage{amsmath,amssymb}
%\usepackage[amssymb]{SIunits}
\usepackage{babel}
\usepackage[latin1]{inputenc}
\usepackage{graphicx}

\title{KOGW-PM-KNP:  Exercise 1 Questions - Runeson`s Planimeter}
\author{}
\date{\today}

\begin{document}

\maketitle

\raggedright
Read the paper on smart perceptual mechanisms by Runeson and answer the following questions. \\


\begin{enumerate}
 \item Explain, in your own words, the difference between perception and cognition, using the concept of a "smart" mechanism [1]? Begin with a basic description of how the planimeter works [1]. Include an example of a discovered smart mechanism [1].
 \item[]
 \item Name two properties that Runeson identifies as definitive for smart mechanisms [2]?
 \item[]
 \item What is the "principle of equal simplicity" [2]?
 \item[]
 \item Name two areas of psychology (besides perception) in which smart mechanisms apply [2]:
 \item[]
 \item Invent a smart mechanism that could improve on our everyday perception [1].
\end{enumerate}

Maximum marks 10/10
\end{document}